\section*{Abstract}
\textbf{Internet of Things (IoT) }has revolutionised how devices communicate together in the 21st century. Today the infrastructure used includes 5G and fibre, allowing faster communication, real-time monitoring and faster data processing. As device types increase daily, many issues arise regarding privacy, automation, ethics, vulnerabilities, consent, and trust. This project aims to develop an ethical framework for using and deploying civilian drones, public smart cameras and briefly explore how cameras are used in autonomous vehicles, with a focus on data governance.
\newline
\vspace{4mm} %5mm vertical space
In my research will identify applicable ethical principles and develop the ethical framework, using a list of questions in the form of a checklist, with particular answers and choices.  The purpose of this framework is to support the “development and deployment” of civilian drones and public cameras informed by ethical principles. I will focus on the ethics related to public smart cameras, civilian drone cameras and autonomous vehicle cameras. Also, a comparison with current frameworks will be made to see if they are compatible and apply to my research area. There are currently many legislations and other codes of practice relating to privacy, I will see how they stand up to the ethical examples and principles. Then delve into the different storage mediums relating to smart camera devices and develop a list of ethical principles for device usage, data management and related software development.
\newline
\vspace{5mm} %5mm vertical space
When it comes to public smart cameras, many considerations need to be considered for the public interest. These considerations include: How and where data is stored?; What are the implications of this technology?; Why is a camera mounted in my street?; What are some of the concerns and challenges faced in the public interest?. 
\newline
\vspace{5mm} %5mm vertical space
Although we hear the term no-fly-zone when using civilian drones, we do not hear much about ethics and privacy. We tend to forget that drones are also big flying cameras that take videos and images. When flying a drone, the pilot needs to consider safety, drone registration, ethical principles, privacy and much more. According to Australian law, taking a picture might be perfectly legal, but using that picture in a particular manner might land you in a legal battle. I will also explore the steps one needs to take to register their drone in Australia. 
\newline
\vspace{5mm} %5mm vertical space
The IoT is based on a high level of security, protocols, authenticity and ethics, people will always defy these systems to gain control of large enterprises. In support of the proposed framework, I will explore the different means and vulnerabilities hackers use to gain control, isolating minor and significant problems relating to specification. Some of these devices might be hijacked using external software, through a connection on the devices i.e. a memory card slot or USB connection, remotely or even damaged physically. 
\newline
\vspace{5mm} %5mm vertical space
Lastly, the paper will focus on two critical factors, and they are hardware and software maintenance. As devices are rolled out, the hardware may need replacement due to damages, such as replacing a lens cover or even a cable. More critical is software maintenance, as this is crucial to all major stakeholders. 
